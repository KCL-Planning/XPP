\documentclass{llncs}
%\documentclass[letterpaper]{article}
%\usepackage{aaai}
\usepackage{times}
\usepackage{helvet}
\usepackage{courier}
%% \setlength{\pdfpagewidth}{8.5in}
%% \setlength{\pdfpageheight}{11in}
\usepackage[usenames]{color} % Only used in comment commands

\usepackage{epsfig,graphics,latexsym}
\usepackage{amsmath,amssymb,enumerate}
\usepackage{array}
\usepackage{wasysym}
\usepackage{multirow}
\usepackage{stmaryrd}
\usepackage{float}
\usepackage{graphicx}
\usepackage{rotating,makecell}
\usepackage{times}
\usepackage{enumerate}
\usepackage{url}
\usepackage{pseudocode}
%\usepackage[usenames]{color} % Only used in comment commands
%\usepackage{caption}
%\usepackage[]{subcaption}
\usepackage{listings}

%\usepackage{amsthm}

\usepackage{epsf}
\usepackage{xy}
\xyoption{all}
\graphicspath{{IMAGES/}}

% \usepackage{pgf}
% \usepackage{tikz}
% \usetikzlibrary{snakes,arrows,shapes,patterns}
% % \usetikzlibrary{arrows,petri,positioning,automata,fit,shapes}
% \tikzstyle{fluent}=[draw,minimum size=.75cm,thick,shape=circle]
% %\tikzstyle{initfluent}=[fluent,double]
% %\tikzstyle{goalfluent}=[fluent,fill=black!10]
% \tikzstyle{goalfluent}=[fluent,double]
% \tikzstyle{initfluent}=[fluent]
% \tikzstyle{graphnode}=[draw,minimum size=0cm,thick,shape=circle,fill=black]
% \tikzstyle{edgelabel}=[minimum size=.35cm,inner sep=0mm,auto]
% \tikzstyle{statenode}=[shape = circle, draw, minimum size = .75cm]
% \tikzstyle{every edge}+=[thick]


%definition environment
%\newtheorem{mydef}{Definition}[chapter]
%\newtheorem{myexp}{Example}[chapter]
%\newtheorem{mytheo}{Theorem}[chapter]

\newcolumntype{L}[1]{>{\raggedright\arraybackslash}p{#1}}
\newcolumntype{C}[1]{>{\centering\arraybackslash}p{#1}}
\newcolumntype{R}[1]{>{\raggedleft\arraybackslash}p{#1}}

\definecolor{mygreen}{HTML}{009933}

\newcommand{\abseq}{/\hspace{-0.18cm}\sim^\alpha}

\newcommand{\titledate}[2][2.5in]{%
  \noindent%
  \begin{tabular}{@{}p{#1}@{}}
    \\ \hline \\[-.75\normalbaselineskip]
    #2
  \end{tabular}
}

\renewcommand{\textfraction}{0.05}



%\setcounter{secnumdepth}{0}

%%%%%%%%%%
% PDFINFO for PDFTEX
% Uncomment and complete the following for metadata
% (your paper must compile with PDFTEX)
\pdfinfo{
/Title (Explainable AI Planning (XAIP): Overview and the Case of Contrastive Explanation (Extended Abstract))
/Author (Joerg Hoffmann and Daniele Magazzeni)
/Keywords (AI Planning, Explainable AI)
}
%%%%%%%%%%

\begin{document}

%\nocopyright

\title{Explainable AI Planning (XAIP): Overview and the Case of
  Contrastive Explanation (Extended Abstract)}

\author{J\"org Hoffmann\inst{1} \and Daniele Magazzeni\inst{2}}

\institute{Saarland Informatics Campus, Saarland University\\
Saarbr\"ucken, Germany\\
hoffmann@cs.uni-saarland.de\and
King's College London\\
London, UK\\
daniele.magazzeni@kcl.ac.uk
}

\maketitle

% abstract as sent to RW:

%% Explainable AI Planning (XAIP): Overview and the Case of
%% Contrastive Explanation

%% Model-based approaches to AI are well suited to explainability in
%% principle, given the explicit nature of their world knowledge and
%% of the reasoning performed to take decisions. AI Planning in
%% particular is relevant in this context as a generic approach to
%% action-decision problems. Indeed, explainable AI Planning (XAIP)
%% has received interest since more than a decade, and has been taking
%% up speed recently along with the general trend to explainable AI.

%% The lecture offers an introduction to the nascent XAIP area. The
%% first half provides an overview, categorizing and illustrating the
%% different kinds of explanation relevant in AI Planning, and placing
%% previous work in this context. The second half of the lecture goes
%% more deeply into one particular kind of XAIP, contrastive
%% explanation, aimed at answering user questions of the kind "Why do
%% you suggest to do A here, rather than B (which seems more
%% appropriate to me)?". Answers to such questions take the form of
%% reasons why A is preferrabe over B. Covering recent work by the
%% lecturers towards this end, we set up a formal framework allowing
%% to provide such answers in a systematic way; we instantiate that
%% framework with the special case of questions about goal-conjunction
%% achievability in oversubscription planning (where not all goals can
%% be achieved and thus a trade-off needs to be found); and we discuss
%% the compilation of more powerful question languages into that
%% special case. Linking to the state of the art in research on
%% effective planning methods, we briefly cover recent techniques for
%% nogood learning in state space search, as a key enabler to
%% efficiency in the suggested analyses.

\begin{abstract}
Model-based approaches to AI are well suited to explainability in
principle, given the explicit nature of their world knowledge and of
the reasoning performed to take decisions. AI Planning in particular
is relevant in this context as a generic approach to action-decision
problems. Indeed, explainable AI Planning (XAIP) has received interest
since more than a decade, and has been taking up speed recently along
with the general trend to explainable AI. In the lecture, we provide
an overview, categorizing and illustrating the different kinds of
explanation relevant in AI Planning; and we outline recent works on
one particular kind of XAIP, contrastive explanation. This extended
abstract gives a brief summary of the lecture, with some literature
pointers. We emphasize that completeness is neither claimed nor
intended; the abstract may serve as a brief primer with literature
entry points.
\end{abstract}



\section{Explainable AI Planning: Overview}

\joerg{todo Dan then Joerg: spell out, cite literature.}

The first half provides an overview, categorizing and illustrating the
different kinds of explanation relevant in AI Planning, and placing
previous work in this context.

\joerg{relevant snippets from IJCAI intro:}

Explainable AI (XAI) is concerned with making AI systems' decisions
more lucid and thus trustworthy. AI planning is relevant to XAI as a
decision-making methodology, model-based and thus suited to provide
explanations in principle. Consequently, research on explainable AI
planning (XAIP) has received increasing interest in recent years
(\eg\ \cite{seegebarth:etal:icaps-12,smith:aaai-12,langley:etal:aaai-17,fox:etal:ijcai-ws-17,chakraborti:etal:ijcai-17,chakraborti:icaps-19}).


\joerg{relevant snippets from AFOSR proposal:}

The need for explainable AI (XAI) first became prominent in Machine
Learning, where the lack of understandable decision rationales is
particularly daunting. Model-based techniques are fundamentally better
suited to providing explanations, yet also there explainability has
traditionally not been considered, and such consideration is currently
attracting increased interest. One culminating point has been the
IJCAI-17 XAI workshop, whose focus was mainly ML but that also started
to reach out to other areas of AI
already.\footnote{http://home.earthlink.net/$\sim$dwaha/research/meetings/ijcai17-xai/}
Daniele Magazzeni is very active in this nascent community; he has
given invited talks on the subject at the IJCAI'17 XAI workshop, at
the ICAPS'17 Scheduling and Planning Applications workshop, at the
ICAPS'17 Planning and Robotics workshop, as well as at the ICAPS'17
User Interfaces and Scheduling and Planning workshop.

Most previous works on XAIP aim at helping humans to understand the
inner workings of a plan suggested by the system (\eg,
\cite{mcguiness:etal:flairs-07,khan:etal:icaps-09,bidot:etal:mkwi-10,sohrabi:etal:aaai-11,seegebarth:etal:icaps-12,bercher:etal:icaps-14,nothdurft:etal:sigdal-15}). This
involves, in particular, the transformation of planner output (\eg,
PDDL plans) into forms that humans can easily understand; the
description of causal and temporal relations between individual plan
steps; and the design of interfaces, in particular suitable dialogue
systems, supporting human interaction and understanding.

Explaining the inner workings of one particular plan is quite
different from our aim of explaining the space of all possible
plans. The only previous work in the latter direction, by
\cite{goebelbecker:etal:icaps-10}, addresses the special case where
there is no plan at all: a so-called \emph{excuse} is a minimal
modification that would render the task solvable (as in ``I would have
been punctual had the bus arrived as scheduled''). Clearly, this does
not address the fine-grained comparison of different plans, and the
associated trade-offs, in solvable tasks.

Manuela Veloso and her team investigate verbalizations describing the
robot experience and intentions to human users
\cite{rosenthal:etal:ijcai-16}. Some recent work
\cite{zhang:etal:icra-17} focuses on a human's interpretation of
plans. Learning is used to create a model of the interpretation, which
is then used to measure the explicability and predictability of
plans. In \emph{model
  reconciliation}~\cite{chakraborti:etal:ijcai-17}, the focus is on
the agent and the human having two different world models. The
explanation must then identify and reconcile the relevant differences
between these models. \cite{langley:etal:aaai-17} coined the term
\emph{explainable agency} to refer to the ability of autonomous agents
to explain their decisions. They discuss some functions that agents
should exhibit.

Our approach here is most closely related to, and inspired by, the
challenges set by David Smith in what he called \emph{Planning as an
  Iterative Process} \cite{smith:aaai-12}. This introduces a broad
vision of users interacting with the planning process, to elaborate
their preferences, to understand the planner's decision rationales, to
interactively produce the final plan. A similar vision of plan
explanation was also proposed in recent work by Daniele Magazzeni and
co-authors \cite{fox:etal:ijcai-ws-17}, as one of out of several
relevant challenge scenarios. Our proposal can be viewed as a
technical approach promising to realize these visions.







\section{Contrastive Explanation}

\joerg{todo Dan then Joerg: spell out, cite literature.}

The second half of the lecture goes more deeply into one particular
kind of XAIP, contrastive explanation, aimed at answering user
questions of the kind "Why do you suggest to do A here, rather than B
(which seems more appropriate to me)?". Answers to such questions take
the form of reasons why A is preferrabe over B.

\joerg{relevant snippets from IJCAI intro:}

A recent analysis \cite{miller:ai-19} of lessons to be learned for XAI
from social sciences highlights that user questions are often
\emph{contrastive}. A question ``Why this?'' actually means ``Why this
\emph{rather than something else} that I would expect?''. To address
such queries, explanatory systems should analyse alternative
solutions, and support the user in understanding the consequences of
the ``something else'' in question.
%
AI planning fits well for this kind of analysis. Two prior works
designed variants thereof
\cite{fox:etal:ijcai-ws-17,miller:corr-18}. The work by Fox et al.\ is
the starting point of our work here.

Fox et al.\ suggest, given a plan $\plan$ and a user question ``Why
does $\plan$ start with action $A$ rather than $B$?'', to generate a
new plan $\plan'$ starting with $B$, and answer the question based on
comparing the two plans: undesirable properties of $\plan'$ serve to
explain the previous decision. While this idea is natural, a key
weakness is
%
%%the potentially arbitrary nature of $\plan'$.
%
%% One difficulty is that the planner might choose to
%% simply undo $B$ and re-insert $A$. More generally, t
%
that there may be differences between \plan\ and $\plan'$ unrelated to
the use of $A$ vs.\ $B$. Many comparison aspects (\eg\ which other
actions are used, or which ``soft'' objectives are satisfied) may be
affected by arbitrary decisions in the planner's search.




\section{Contrastive Explanation of Plan Space through Plan-Space Dependencies}
\label{xpp}

\joerg{todo Joerg then Dan: spell out, cite literature.}

Covering recent work by the lecturers towards this end, we set up a
formal framework allowing to provide such answers in a systematic way;
we instantiate that framework with the special case of questions about
goal-conjunction achievability in oversubscription planning (where not
all goals can be achieved and thus a trade-off needs to be found); and
we discuss the compilation of more powerful question languages into
that special case. Linking to the state of the art in research on
effective planning methods, we briefly cover recent techniques for
nogood learning in state space search, as a key enabler to efficiency
in the suggested analyses.

\joerg{relevant snippets from IJCAI intro:}


Our approach here is most closely related to, and inspired by, the
challenges set by David Smith in what he called \emph{Planning as an
  Iterative Process} \cite{smith:aaai-12}. This introduces a broad
vision of users interacting with the planning process, to elaborate
their preferences, to understand the planner's decision rationales, to
interactively produce the final plan. 



Here we address the same kind of explanation problem, but we replace
the \emph{existential} answer generating a single alternative plan
$\plan'$ with a \emph{universal} answer determining shared properties
of \emph{all} possible such alternatives. In this way, the analysis we
propose aims at explaining the space of possible plans, rather than
pointing out examples.

Our proposed analysis works at the level of \defined{plan properties}:
Boolean functions on plans that capture aspects of plans the user
cares about (whether or not the plan starts with a particular action,
whether or not a particular soft objective is satisfied, etc). We
assume that the set \props\ of plan properties of interest is given as
part of the input.\footnote{An interesting yet challenging question
  for future work is how one can automatically identify relevant plan
  properties.} Our analysis then determines the \defined{dependencies}
across plan properties, \ie, \defined{plan-space entailments} which we
define as follows. The ``plan space'' is the set \plans\ of candidate
plans to be considered (canonically, the set of plans for an input
planning task). A plan property $p$ \defined{entails} another property
$p'$ in \plans\ if every $\plan \in \plans$ that satisfies $p$ also
satisfies $p'$. A user question ``Why does the current plan
\plan\ satisfy $p$ rather than $q$?'' can then be answered in terms of
the properties $q'$ not true in \plan\ but entailed by $q$: things
that will \emph{necessarily} change when satisfying $q$.

Our approach also supports iterative planning, along the lines
suggested by Smith \cite{smith:aaai-12}. Given a current plan
$\plan \in \plans$ and a user question ``Why achieve $p$ rather than
$q$?'', if the consequences of $q$ are tolerable to the user, she may
choose to enforce $q$, gradually narrowing the plan-candidate space
\plans.
%
% Joerg: Text highlighting enforced vs analyzed; simplified to save space
%
%% Observe that \plans\ itself may be viewed as being defined through a
%% set of \emph{enforced} plan properties (like achieving a set of goal
%% facts). Such enforced properties are then distinguished from the
%% \emph{analyzed} properties whose dependencies we wish to identify.
%% %
%% These two classes of properties can play different roles depending on
%% the application scenario. In contrastive explanations as outlined
%% above, the enforced properties are fixed. However, our approach also
%% supports an iterative planning process for oversubscription planning
%% (\eg\ \cite{smith:icaps-04,domshlak:mirkis:jair-15}), along the lines
%% suggested by Smith \cite{smith:aaai-12}. The analyzed properties
%% then capture ``soft goals'', while the enforced properties capture
%% ``hard goals''. Given a currently suggested plan $\plan \in \plans$
%% and a user question ``Why $p$ rather than $q$?'', if the consequences
%% of analyzed property $q$ are tolerable to the user, she may choose to
%% enforce $q$, gradually narrowing the plan-candidate space \plans.
%% %
%% % Joerg: shortetened to save space
%% %
%% %% Observe that \plans\ itself may be naturally defined as the set of
%% %% plans satisfying a given set of plan properties. For example, these
%% %% properties may ask to achieve a set of goal facts. In such a setting,
%% %% it makes sense to distinguish between \defined{enforced} plan
%% %% properties, that induce \plans; vs.\ \defined{analyzed} plan
%% %% properties, whose entailment relations within \plans\ we wish to
%% %% identify. 
%% %
%% %% Enforced vs.\ analyzed properties can play different roles depending
%% %% on the application scenario. In classical planning, the analyzed
%% %% properties may capture relevant plan phenomena in a user quest to
%% %% understand causal relationships between these phenomena
%% %% (\eg\ dependencies between action subsets used). Another use case is a
%% %% user quest to identify a preferred plan in oversubscription planning
%% %% (\eg\ \cite{smith:icaps-04,domshlak:mirkis:jair-15}), where the
%% %% analyzed properties capture ``soft goals'', and the enforced
%% %% properties are ``hard goals''. The analysis then identifies the
%% %% precise trade-offs between the soft goals.
%% %% %
%% %% % Joerg: too complicated/more distracting than helpful
%% %% %
%% %% %% ; one may include additional analyzed properties aimed at identifying
%% %% %% the causes behind these trade-offs.
%% %% %
%% %% In that setting, our approach also supports an iterative planning
%% %% process along the lines suggested by Smith \cite{smith:aaai-12}:
%% %% given a currently suggested plan $\plan \in \plans$ and a user
%% %% question ``Why $p$ rather than $q$?'', if the consequences of analyzed
%% %% property $q$ are tolerable to the user, she may choose to enforce $q$,
%% %% gradually narrowing the candidate space \plans.

We remark that our approach can be viewed as an intermediate between
domain/task analysis (\eg\ \cite{fox:long:jair-98}), which our
approach generalizes; and model checking applied to planning models,
which our approach is an instance of (related to
\cite{vaquero:etal:keq-13}). 
%
% Joerg: Detailed discussion of domain analysis and model checking;
% simplified to save space/not be distracting here.
%
%% Another alternate view of our approach is as a form of domain analysis
%% (actually: task analysis), identifying particular properties of plan
%% space ahead of time. Indeed, various popular task analyses can be cast
%% as instances of our framework. A fact pair $(p,q)$ is mutually
%% exclusive \cite{blum:furst:ai-97} iff $p$-true-at-end entails $\neg
%% q$-true-at-end in the space of all applicable action sequences; a fact
%% $p$ is a landmark \cite{hoffmann:etal:jair-04} iff $\true$ entails
%% $p$-true-at-some-point; other examples presumably exist. From this
%% point of view, we generalize previous concepts to a broader
%% perspective aimed at addressing arbitrary user questions. At the same
%% time, our approach itself can be viewed as an instance of model
%% checking of planning models \cite{clarke:etal:01},\footnote{There has
%%   been little work on this subject; Vaquero et
%%   al.\ \cite{vaquero:etal:keq-13} use Petri nets to capture and
%%   check dynamic aspects of planning models in itSIMPLE.}
%% systematically checking all entailments between plan properties. Again
%% the value of our framework lies in its suitability for XAIP (plus
%% computational gains from considering all \props\ dependencies in
%% unison rather than running individual entailment checks).





\subsubsection*{Acknowledgments}

This material is based upon work supported by the Air Force Office of
Scientific Research under award number FA9550-18-1-0245. J\"org
Hoffmann's research group has received support by DFG grant 389792660
as part of TRR~248 (see \url{https://perspicuous-computing.science}).

\bibliographystyle{plain}
\bibliography{../BIBLIO/abbreviations,../BIBLIO/biblio,../BIBLIO/crossref}

\end{document}
