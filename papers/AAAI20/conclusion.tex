\section{Conclusion}
\label{conclusion}

%% \joerg{0.1--25 page: adapt ijcai version.}


Our approach analyses plan space in terms of plan properties and their
dependencies. This naturally addresses challenges raised in previous
work on iterative planning, and performance is reasonable in IPC
benchmark studies. The approach can be easily generalized to almost
arbitrary planning frameworks and plan-property languages, and we hope
it will inspire other researchers too. To name just one example, plan
properties and their dependencies might help in plan verbalization
\cite{rosenthal:etal:ijcai-16}, as an abstraction level (plan
properties: ``I did X'') as well as a means for justification
(dependencies: ``I did not do Y because \dots'').

An interesting challenge is to answer deeper ``why'' questions
regarding an entailment $\entails{\plans}{\prop}{\propq}$. Possible
ideas are to include additional plan properties elucidating the causal
chain between \prop\ and \propq; or to find a minimal relaxation
(superset) of the plan set \plans\ for which \prop\ no longer entails
\propq, thus elucidating the circumstances under which that entailment
holds. A computational idea worth trying is to represent the plan set
\plans\ symbolically and use that representation to identify
entailment relations. We aim to apply and evaluate our approach in
concrete use cases such as simulated penetration testing and AUV
mission control \cite{cashmore:etal:icra-14}.
%
% Joerg: better just not go there ...
%
%Such applications will also enable meaningful user evaluations.

% Joerg: reshaped this
%
%% In future work, the effectiveness of these explanations for human
%% users remains to be evaluated in user studies. Another important
%% question is how to address deeper ``why'' questions, asking for the
%% reasons behind an entailment
%% $\entails{\plans}{\prop}{\propq}$. Possible ideas are to include
%% additional properties into \props, elucidating the causal chain
%% between \prop\ and \propq; or to find a minimal relaxation
%% (superset) of the plan set \plans\ for which \prop\ no longer
%% entails \propq, thus elucidating the circumstances under which that
%% entailment holds. Last but not least, of course our framework and
%% algorithms can and should be extended to richer planning frameworks
%% and plan property languages.


% Joerg: this is rather from our stuff but I think I get where he's
% coming from. Basically, plan properties can be viewed as the
% abstraction level for the explanation (``my plan has properties x y
% z''. This does not really get rid of templates, it's more like
% replacing the templates with plan properties. The plan-property
% dependencies could serve as a new feature in the explanations,
% elucidating why the robot chose to do X rather than Y (I havent seen
% anything of the sort on their work, though I might ave overlooked
% it, didnt study this closely).
%
%% Comment: I was thinking about the plan verbalization work from Manuela Veloso's group 
%% -- e.g. \url{https://www.ijcai.org/Proceedings/16/Papers/127.pdf} 
%% -- and realized there may be some interesting connections here. 
%% Granted these are different kinds of plans, motion plans versus plans in general, 
%% but...  I wonder if an approach like in this paper can be a better alternative 
%% to the template approach in the verbalization work. For example, a PDO versus 
%% a concrete PDO could help with "specificity" while planning models do have 
%% a rich literature of abstractions that can be readily leveraged in this framework. 
%% I guess the point is if it is useful automating the templatization using plan properties. 
%% Maybe something to think about... 

%% More (recent) details here if you are interested:
%% \begin{enumerate}
%% 	\item \url{http://www.cs.cmu.edu/~mmv/papers/18gcai-PereraVeloso.pdf}
%% 	\item (data you can use) \url{http://www.cs.cmu.edu/~vdperera/paper/vdperera_thesis.pdf#page=129}
%% \end{enumerate}

