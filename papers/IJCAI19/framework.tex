\section{Generic Framework}
\label{framework}

% Joerg: not needed anymopre, said in intro
%
%% We first introduce our approach making minimal assumptions on the
%% planning context and plan properties concerned. The framework can
%% then be instantiated with concrete planning formalisms and
%% plan-property languages of interest, as we will exemplify in later
%% sections. The framework uses standard notions from logics, and its
%% main intention and contribution is to conceptualize the
%% explainability problems we address.

We assume some formalism defining planning \defined{tasks} \task. We
do not need any assumptions about that formalism, except that it
defines a concept of \defined{plans} \plan, where that concept can
again be arbitrary (action sequence/schedule/partial order/etc). 
%
Our definitions are relative to a set \plans\ of plans of
interest. The canonical setup we have in mind is that where \plans\ is
induced by \task, as the set of action sequences applicable in the
initial state, or as the set of plans that achieve a goal. It could
also be useful in some cases though to focus the analysis on a small
set of candidate plans listed as part of the input.


% Joerg: Text highlighting enforced vs analyzed; simplified to save space
%
%% \subsection{Planning Tasks and Plans}

%% We assume some formalism defining planning \defined{tasks} \task. We
%% do not need any assumptions about that formalism, except that it
%% defines a concept of \defined{plans} \plan, where that concept can
%% again be arbitrary (action sequence/schedule/partial order/etc). 

%% Our definitions are relative to a set of plans \plans\ of
%% interest. The canonical setup we have in mind is that where \plans\ is
%% \defined{implicit}, induced by \task: \plans\ may be the set of action
%% sequences applicable in the initial state; or it may be the set of
%% plans which satisfy a set of \defined{enforced} plan properties (and,
%% potentially, some optimality criterion). But if the user wants to
%% focus the analysis on a narrow/small subset of candidate plans, then
%% \defined{explicit} \plans, given as a list of plans in the input, can
%% also be relevant. In either case, our framework addresses dependencies
%% between \defined{analyzed} plan properties within \plans.
%% %
%% % Joerg: previous text here
%% %
%% %% Our definitions are relative to a set of plans \plans\ of interest:
%% %% the set of plan \emph{candidates} the human user is considering, and
%% %% whose property dependencies we should thus analyze. The canonical
%% %% setup we have in mind is that where \plans\ is \defined{implicit},
%% %% induced by \task\ as the set of \plan\ that achieve a set of
%% %% \defined{hard} goals/constraints (in general: \defined{plan
%% %%   properties}, see below) which the user is not willing to forego
%% %% under any circumstance. But \defined{explicit} \plans, given as a list
%% %% of plans in the input, can also be relevant if the user wants to focus
%% %% the analysis on a narrow/small subset of candidate plans. Whichever is
%% %% the case, the analysis considers dependencies between \defined{soft}
%% %% plan properties within \plans. These properties can be soft goals as
%% %% in oversubscription planning, but they can also be arbitrary
%% %% constraints characterizing the manner in which the hard plan
%% %% properties are achieved. For example, in classical planning,
%% %% \plans\ can simply be the set of (cost-optimal) plans achieving the
%% %% goal, where our analysis answers questions about the properties of
%% %% such plans.





\subsection{Plan Properties and Property Entailment}

Plan properties, in their most general form, are simply functions
mapping a task and plan to a Boolean value indicating whether or not
the property is satisfied:

\begin{definition}[Plan Property]
Denoting by \alltasks\ the set of all tasks and by \allplans\ the set
of all plans, a \emph{plan property} is a partial function $\prop :
\alltasks \times \allplans \mapsto \{\true, \false\}$. Given a task
\task\ and a set of plans \plans, we say that \prop\ is a plan
property defined on \task\ and \plans\ if its domain includes
$\{(\task,\plan) \mid \plan \in \plans\}$.
\end{definition}

Example plan properties are goal facts/goal formulas (true at end of
plan?), temporal plan trajectory constraints, constraints on subsets
of actions used/not used, deadlines, bounds on resource consumption,
etc. We expect that, typically, $\prop$ will be computable in time
polynomial in the size of its input (though that is not a requirement
of our framework).

We assume a set \props\ of plan properties as part of our input. That
set may be exponentially large in the size of its specification. An
example we will explore later is that where the user is interested in
dependencies between subsets of a set $G$ of soft-goal facts. The set
\props\ of interest then are the conjunctions $\phi$ over $G$
(functions checking whether $\phi$ is true at the end of a plan), but
the input to our analysis specifies only $G$.
%
% Joerg: Text highlighting atomic vs composed; simplified to save space
%
%% We assume a set \props\ of plan properties (the analyzed ones) as part
%% of our input. In this context, a distinction not necessary for our
%% generic definitions, but important in practice, is that between
%% \defined{atomic} vs.\ \defined{composed} plan properties. The set
%% \propsatom\ of atomic properties is listed explicitly in the input,
%% whereas the composed properties \propscomp\ arise from a compact
%% specification how properties can be combined. For example, one may
%% define \propscomp\ to be propositional formulas over the atoms
%% \propsatom. Note that, given this, the set $\props = \propsatom \cup
%% \propscomp$ of plan properties whose dependencies are being analyzed
%% may be exponentially larger than the user input, and may even be
%% infinite.
%
%% As a concrete instance we will consider later on, the user may be
%% interested in a set $G$ of soft-goal facts, where, for each $g \in G$,
%% being or not being achieved by a plan $\plan \in \plans$ is an atomic
%% plan property. But the interesting dependencies may be, not across
%% individual $g$, but over conjunctions thereof (\eg\ if conjunctions,
%% but not typically singletons, exclude the possibility to achieve other
%% goals). The input to our analysis then is the set $G$ (atomic,
%% \propsatom), while the analysis considers dependencies across all
%% conjunctions over $G$ (composed, \propscomp).

The kind of dependency our framework focusses on is entailment over
plan properties, in the space of truth-value assignments induced by
the plan-candidate set \plans:

%% \joerg{I deliberated adapting the following definitions and text to
%%   explicitly use two sets of plan properties, enforced vs analyzed,
%%   where the enforced ones replace the role currently taken by
%%   \plans. This would be nice in that it makes visible how properties
%%   can move between these two sides, and that entailment in the KB is
%%   the same, then, as moving the required properties from the KB into
%%   the LHS of an implication which is then valid in the reduced KB --
%%   overall, this yields a nice match with the usual stuff one is used
%%   to in logics. However, the definitions as stated here are 1)
%%   strictly more general, as they allow such a setup in particular but
%%   also allow explicitly given \plans; 2) better suited to the case
%%   where \plans\ has optimality criteria as well, which can be captured
%%   as plan properties but odd ones not computable in polynomial time;
%%   3) better suited to our discussion here as they focus on the issue
%%   of entailment in plan space, where the set of plans is a KB, which
%%   is our main focus and new thing here.}

%% \joerg{mention task properties? (ini state etc)? decided to not do
%%   this here: 1) these make most intuitive sense when explcitly making
%%   them enforced properties, that define the space of plans \plans\ we
%%   are interested in; 2) it's rather a side line here, and distracting
%%   from the main focus of our discussion introducing the framework,
%%   which already covers many aspects and can be confusing.}

\begin{definition}[\plans-Entailment]
Let \task\ be a task, \plans\ a set of plans, and \props\ a set of
plan properties defined on \task\ and \plans.

Let $\plan \in \plans$. We identify \plan\ with the truth-value
assignment $\plan : \props \mapsto \{\true, \false\}$ where
$\plan(\prop) := \prop(\task,\plan)$. We identify \plans\ with the set
of such truth-value assignments. We say that
\plan\ \defined{satisfies} \prop, written $\plan \models \prop$, if
$\plan(\prop) = \true$. We denote by $\modelsof{\plans}{\prop} :=
\{\plan \mid \plan \in \plans, \plan \models \prop\}$ the models of
\prop. 
%
%% We say that \prop\ is \defined{satisfiable} in \plans\ if
%% $\modelsof{\plans}{\prop} \neq \emptyset$.

We say that \prop\ \defined{\plans-entails} \propq, written
$\entails{\plans}{\prop}{\propq}$, if $\modelsof{\plans}{\prop}
\subseteq \modelsof{\plans}{\propq}$.
%
We say that \prop\ and \propq\ are \defined{\plans-equivalent},
written $\iff{\plans}{\prop}{\propq}$, if $\modelsof{\plans}{\prop} =
\modelsof{\plans}{\propq}$. We denote $\equiv{\plans}{\prop} :=
\{\propq \mid \propq \in \props, \iff{\plans}{\prop}{\propq}\}$.
\end{definition}

This definition essentially just views plans $\plan \in \plans$ as
truth-value assignments in the obvious manner. Entailment and
equivalence over plan properties are then defined straightforwardly,
with \plans\ in the role traditionally taken by a knowledge base that
restricts the truth-value assignments under consideration. Observe
that formulas over plan properties can be encoded as individual plan
properties, so that defining \plans-entailment over individual plan
properties is enough to permit logical combinations thereof.

Importantly, the role of \plans\ as a knowledge base means that
\plans-entailment is more than standard entailment: the latter implies
the former, but not vice versa. As a simple example, say the plan
properties \props\ are propositional formulas $\phi$ over facts,
evaluated at the end of the plan. Then $\phi \Rightarrow \psi$ implies
that $\entails{\plans}{\phi}{\psi}$, simply because any (plan-end)
state that satisfies $\phi$ must satisfy $\psi$. But not vice versa:
\eg\ if facts $p, q$ are mutex in the task then
$\entails{\plans}{p}{\neg q}$. As a more motivating example, say the
plan properties are goals (like having scientific observations in
satellite planning) as well as resource constraints (like consuming at
most a given amount of energy). Then entailments of interest can take
the form $\entails{\plans}{p}{\neg (q_1 \wedge q_2 \wedge q_3)}$
saying that we cannot have $p$ without foregoing either of $q_1$ or
$q_2$ or $q_3$. Note that this is an entailment specific to \plans,
which may not hold in general (\eg\ if cheaper actions are available,
or if cheaper plans are admitted by removing some other enforced
goals). The identification of such specific entailments -- specific to
the space \plans\ of plans considered -- is central to our framework.





\subsection{Plan-Space Explanations}

Our plan-space explanations identify the \plans-entailment relation on
\props\ given the knowledge base \plans:

\begin{definition}[PDO, cPDO]
Let \task\ be a task, \plans\ a set of plans, and \props\ a set of
plan properties defined on \task\ and \plans.

The \defined{plan-property dependency order (PDO)} for \plans\ and
\props\ is the partial order $\pdo{\plans}$ over the equivalence
classes $\equiv{\plans}{\prop}$, where $\equiv{\plans}{\prop}
\pdo{\plans} \equiv{\plans}{\propq}$ iff
$\entails{\plans}{\prop}{\propq}$.

A \defined{concrete PDO (cPDO)} replaces each equivalence class
$\equiv{\plans}{\prop}$ with exactly one $\prop \in
\equiv{\plans}{\prop}$.
\end{definition}

The PDO explains the plan space \plans\ by making explicit how the
plan properties depend on each other -- saying things like ``we cannot
have $p$ without foregoing either of $q_1$ or $q_2$ or $q_3$''. The
PDO is exhaustive, and in that sense ideal. Its size may be
problematic though. A concrete PDO can serve as a practical proxy if
equivalence classes are large. Beyond that, it is clearly important to
identify (i) more compact and/or (ii) more restricted plan-space
explanations. We introduce variants of both here.

Regarding (i), in our concrete instantiation of this framework we use
\defined{subsumption} over \plans-entailment relations, relying on an
easy-to-test sufficient criterion for \plans-entailment:
%
% Joerg: Text highlighting atomic vs composed; simplified to save space
%
%% The PDO explains the plan space \plans\ by making explicit how the
%% analyzed plan properties depend on each other -- saying things like
%% ``we cannot have $p$ without foregoing either of $q_1$ or $q_2$ or
%% $q_3$''. The PDO is exhaustive, and in that sense ideal. Its size may
%% be problematic though. The number of equivalence classes is small if
%% one is interested in atomic plan properties only. But if
%% \props\ includes composed properties then that number may be
%% prohibitively large. Each individual equivalence class may also be
%% large, or even infinite.
%
%% A concrete PDO can serve as a practical proxy if equivalence classes
%% are large. Beyond that, it is clearly important to identify (i) more
%% compact and/or (ii) more restricted plan-space explanations. We
%% introduce variants of both here. Regarding (i), in our concrete
%% instantiation of this framework we use \defined{subsumption} over
%% \plans-entailment relations, relying on an easy-to-test sufficient
%% criterion for \plans-entailment:

\begin{definition}[Dominant cPDO]\label{def:dcpdo}
Let \task\ be a task, \plans\ a set of plans, and \props\ a set of
plan properties defined on \task\ and \plans.

Let $\entailsuff \subseteq \props \times \props$ be such that, if $\prop
\entailsuff \propq$, then $\entails{\plans}{\prop}{\propq}$.
%
In a cPDO, we say that $\prop \pdo{\plans} \propq$ \defined{subsumes
  $\prop' \pdo{\plans} \propq'$ given $\entailsuff$} if $\prop'
\entailsuff \prop$ and $\propq \entailsuff \propq'$.

A \defined{dominant cPDO (dcPDO)} for \plans\ and \props\ given
$\entailsuff$ is the subset of non-subsumed $\prop \pdo{\plans} \propq$
in a cPDO.
%
%% set $\Psi$ of ordered pairs $\prop \pdo{\plans} \propq$ where there
%% exists a cPDO $\Phi$ such that $\Phi \subseteq \Psi$ is the subset of
%% $\prop \pdo{\plans} \propq$ in $\Phi$ not subsumed by any other
%% $\prop' \pdo{\plans} \propq'$ in $\Phi$ given $\entailsuff$.
\end{definition}

An entailment $\prop \pdo{\plans} \propq$ subsumes another one $\prop'
\pdo{\plans} \propq'$ if its left-hand side is weaker ($\prop'
\entailsuff \prop$) and its right-hand side is stronger ($\propq
\entailsuff \propq'$): in this case, $\prop' \pdo{\plans} \propq'$
follows from $\prop \pdo{\plans} \propq$. A dominant cPDO thus selects
only the strongest \plans-entailments in a cPDO, as a more compact
representation of the information present in that cPDO.%
%
%% For example, if entailments over conjunctions are considered, then the
%% strongest entailments have small conjunctions on the left-hand side
%% and large conjunctions on the right-hand side.

The role of \entailsuff\ here is to qualify the amount of information
we are allowed to use in identifying this compact representation. This
is important because, if we show compacted information to a user, then
the user should be able to de-compact this information -- to obtain
whichever information the user is actually interested in --
effortlessly. A simple restriction is for \entailsuff\ to be
computable in polynomial time, but cognitive abilities may necessitate
stronger restrictions. Here we will consider goal-fact conjunctions
and disjunctions, and use the trivial \entailsuff\ where larger
conjunctions are stronger while larger disjunctions are weaker.

% Joerg: save space ... also, this issue is a bit itchy, as in our def
% we have a form of information loss, where the individual formulas in
% the cPDO pretend to stand for the entire equivalence class while the
% user can cognitively deal only with entailsuff ... probably wiser to
% discuss this only if the reviewers are bothered by it.
%
%% We remark that an alternative to Definition~\ref{def:dcpdo} would
%% be to consider a dominant PDO at the level of equivalence classes
%% $\equiv{\plans}{\prop}$. This alternative incurs complications as
%% equivalence under $\entailsuff$ is not the same as equivalence
%% under full \plans-entailment, and \entailsuff\ would have to be
%% transitive which may be problematic for ``cognitively easy''
%% criteria.
%
%% Nevertheless, dominant PDOs may be preferrable to dominant cPDOs in
%% some application scenarios.

As a simple form of (ii) more restricted plan-space explanations, we
will employ the restriction of focus to a predefined subset \deps\ of
dependencies of interest:

\begin{definition}[Restricted (dc)PDO]
Let \task\ be a task, \plans\ a set of plans, and \props\ a set of
plan properties defined on \task\ and \plans.

Let $\deps \subseteq \props \times \props$ be any binary relation on
plan properties. Then a \defined{(dc)PDO for $\deps$} results from
ignoring \plans-entailments $\entails{\plans}{\prop}{\propq}$ where
$(\prop,\propq) \not\in \deps$.
\end{definition}

Some words are in order regarding complexity. Testing
\plans-entailment encompasses the plan existence problem even for
extremely simple plan properties (asking whether the plan achieves a
fact $p$). This is exacerbated by the size of the PDO. Certainly, a
(dc)PDO should ideally be computed offline, prior to interaction with
a user. 
%
% Joerg: repetition, not really needed here. also, really the picture
% is complicated and this pitch incurs the risk of wrong expectations.
%
%% As our experimental results show, however, in a variety of benchmarks,
%% computing a dcPDO has similar scalability as optimal planning.
%
% Joerg: Text highlighting atomic vs composed; simplified to save space
%
%% Some words are in order regarding finiteness and complexity. The PDO
%% is obviously finite if \props\ is, \eg\ if only atomic properties are
%% considered. In the presence of an infinite set of composed plan
%% properties (such as propositional formulas), the number of equivalence
%% classes may still be finite. In particular, that is so if the set of
%% plans \plans\ of interest is finite (\eg\ if the action set is finite,
%% plan schedules are discrete, and plan size can be finitely bounded),
%% as then there is only a finite number of non-equivalent plan
%% properties.
%% %
%% % Joerg: not true because entailsuff is on formulas not equiv classes;
%% % anyway, should be Ok to leave out.
%% %
%% %% When the PDO is finite, minimal elements exist in any \entailsuff\ and
%% %% thus any dcPDO is non-empty.

%% Regarding complexity, if \plans\ is implicitly given, then testing
%% \plans-entailment encompasses the plan existence problem even for
%% extremely simple plan properties (\eg, asking whether the plan
%% achieves some fact $p$).
%% %
%% %% Regarding complexity, if \plans\ is implicitly given, then testing
%% %% plan-property satisfiability encompasses the plan existence problem
%% %% even for extremely simple plan properties (\eg, asking whether the
%% %% plan achieves some fact $p$). Similarly for \plans-entailments
%% %% $\entails{\plans}{\prop}{\propq}$, proving which corresponds to
%% %% proving a planning task unsolvable (there exists no plan that
%% %% satisfies \prop\ but not \propq).
%% %
%% % Joerg: generating all of Pi is not needed necessarily. simpler to
%% % just leave this out.
%% %
%% %% Tractable cases are limited to explicitly given \plans, and planning
%% %% fragments where \plans\ -- the set of all plans -- can be generated in
%% %% polynomial time. 
%% %
%% This is exacerbated by the potentially exponential size of the PDO for
%% composed plan properties. Certainly, a (dc)PDO should ideally be
%% computed offline, prior to interaction with a user. As our
%% experimental results show, however, in a variety of benchmarks,
%% computing a dcPDO has similar scalability as optimal planning.

% Joerg: a fully formal frame here would need to talk about
% compilations, also of planning tasks ... overkill, instead formulate
% what's needed in the concrete instances here when we get there.
%
%% A property we will rely on in the following sections is that sometimes
%% a PDA for a plan-property set $\props^B$ can be easily obtained from
%% that for a plan-property set $\props^A$:
%
%% \begin{definition}[PDA Derivability]
%% Let $\props^A$ and $\props^B$ be sets of plan properties. We say that
%% a PDA for $\props^B$ is \defined{polynomially derivable} from that for
%% $\props^A$ if there exists an algorithm that runs in time polynomial
%% in the size of its input and that, for any task \task\ on which all
%% $\prop \in \props^A \cup \props^B$ are defined, transforms a PDA for
%% $\props^A$ into one for $\props^B$.
%% \end{definition}






%%%%%%%%%%%%%%%%%%%%%%%%%%%%%%%%%%%%%%%%%%%%%%%%%%%%%%%%%%%%%%%%%%%%%%%%
%%%%%%%%%%%%%%%%%%%%%%%%%%%%%%%%%%%%%%%%%%%%%%%%%%%%%%%%%%%%%%%%%%%%%%%%
%%%%%%%%%%%%%%%%%%%%%%%%%%%%%%%%%%%%%%%%%%%%%%%%%%%%%%%%%%%%%%%%%%%%%%%%
%%% VERSION OF Plan-Space Explanations SECTION WITH PDA AND WITHOUT SUBSUMPTION

% PDA motivation: bit unclear, would need to be argued better. why is
% this interesting? maximum solvable goal sets is a useful example but
% also a trivial one with no interesting ''dependencies''

%% \joerg{we use only the PDO later on. remove cPDO and PDA, and simplify
%%   the discussion accordingly? ... can keep cPDO and PDA stuff for
%%   journal paper, if suitable later on ... what's missing though here
%%   is a compact representation of the PDO, not through a PDA, but
%%   instead through showing only the strongest dependencies, where all
%%   other dependencies are subsumed ... so this is the notion we should
%%   insert and discuss here instead? not sure though whether there is a
%%   generic notion of ``subsumption'' that makes sense ... hmwell I
%%   guess one could say that $\entails{\plans}{\prop}{\propq}$ subsumes
%%   $\entails{\plans}{\prop'}{\propq'}$ if
%%   $\entails{\plans}{\prop}{\prop'}$ and
%%   $\entails{\plans}{\propq'}{\propq}$, ie, if its LHS is weaker while
%%   its RHS is stronger. but then, this subsumption relation is not
%%   easily computable by the user, so it is useless in practice. in our
%%   concrete use cases, we instead say that
%%   $\entails{\plans}{\prop}{\propq}$ subsumes
%%   $\entails{\plans}{\prop'}{\propq'}$ if $\prop\Rightarrow\prop'$ and
%%   $\propq'\Rightarrow\propq$, in sub-classes of formulas where
%%   $\Rightarrow$ is easily testable. this makes practical sense; one
%%   could impose a restriction on testability of the entailment relation
%%   used in subsumption ... or we could leave this out of the framework
%%   altogether and merely introduce it later... what I don't like about
%%   this simple solution is that, then, the framework does not contain
%%   an essential aspect of what makes our concrete stuff later on
%%   feasible, namely that the PDO can be represented in terms of a small
%%   set of strongest entailments. TBD.}

%% Our plan-space explanations identify the entailment relation on
%% \props\ given the knowledge base \plans:

%% \begin{definition}[PDO and PDA]
%% Let \task\ be a task and \plans\ a set of plans. Let \props\ be a set
%% of plan properties defined on \task\ and \plans.

%% The \defined{plan-property dependency order (PDO)} for \plans\ and
%% \props\ is the partial order $\pdo{\plans}$ over the equivalence
%% classes $\equiv{\plans}{\prop}$, where $\equiv{\plans}{\prop}
%% \pdo{\plans} \equiv{\plans}{\propq}$ iff
%% $\entails{\plans}{\prop}{\propq}$.
%% %
%% The \defined{plan-property dependency axiomatization (PDA)} for
%% \plans\ and \props\ is the set of minimal elements of $\pdo{\plans}$
%% restricted to \prop\ satisfiable in \plans.

%% A \defined{concrete PDO (cPDO)} (\defined{concrete PDA (cPDA)}) is
%% like the PDO (PDA) but representing each equivalence class
%% $\equiv{\plans}{\prop}$ by exactly one $\prop \in
%% \equiv{\plans}{\prop}$. Given a subset $D \subseteq \props \times
%% \props$, a (c)PDO (PDA) for $D$ results from restricting the
%% entailment relation $\entails{\plans}{\prop}{\propq}$ to
%% $(\prop,\propq) \in \deps$.
%% \end{definition}

%% % Joerg: not really needed
%% %
%% %% We refer by the \defined{PDO problem} to the computational problem of
%% %% computing the PDO given \task\ as well as a (compact) specification of
%% %% \props. Accordingly for the \defined{PDA problem}.

%% The PDO explains plan space by making explicit how the analyzed plan
%% properties depend on each other -- saying things like ``we cannot have
%% $p$ without foregoing either of $q_1$ or $q_2$ or $q_3$''. The PDO is
%% exhaustive, and in that sense the ideal plan-space explanation in
%% principle. In practice though its size, \ie, the number of equivalence
%% classes, may be problematic. If one is interested in atomic plan
%% properties only, specified explicitly in the user input, then
%% \props\ and hence the PDO is small. If \props\ includes composed
%% properties though, then the PDA may be more suited to show as overview
%% information.

%% The PDA captures the strongest properties, that together entail all
%% other properties given \plans, thus forming an axiomatization for
%% \plans. Unsatisfiable properties are not of interest here as they
%% trivially entail everything. 
%% %
%% % Joerg: misleading wrpto what we actually do later 
%% %
%% %% For example, in oversubscription planning, when plan properties are
%% %% conjunctions of soft goals, larger conjunctions entail smaller ones
%% %% and the PDA specifies the maximal solvable conjunctions.

%% %% We remark that the PDO is a meet-semilattice if $\props$ is closed
%% %% under conjunction, \ie, for every satisfiable $p, q \in \props$ there
%% %% is a satisfiable $r \in \props$ which is true if both $p$ and $q$ are
%% %% true. In this case, the PDA is a single plan property ... JOERG:
%% %% whatever. not gonna happen/who cares.
%% %
%% %% \joerg{reg lattices: $O$ is a ``meet-semilattice'' if $\Phi$ is closed
%% %%   under conjunction: in a lattice, every two elements need to have a
%% %%   common ancestor. in standard logic, entailment over formula
%% %%   equivalence classes is a lattice because for any $\phi$ and $\psi$
%% %%   $\phi \wedge \psi$ is a common ancestor; actually a bounded lattice,
%% %%   with a unique element that is an ancestor to (that implies)
%% %%   everything else. the same is true here if $\Phi$ is closed under
%% %%   conjunction (because if each of $\phi$ and $\psi$ is valid, then so
%% %%   is $\phi \wedge \psi$). In this case, the PDA is the unique common
%% %%   ancestor, which can basically be thought of as a conjunction of
%% %%   axioms. In lattice terminology, the unique common ancestor is called
%% %%   a ``least element'', ``minimum'', or ``bottom'' element; but I think
%% %%   none of these names makes much sense in our context so I would leave
%% %%   it at the above definition, maybe briefly remarking here that $O$ is
%% %%   a meet-semilattice if $\Phi$ is closed under conjunction.}

%% A concrete PDO/PDA is a practical proxy if equivalence classes are
%% large or infinite. If only a subset \deps\ of potential entailment
%% relationships are of interest, then a (c)PDO/PDA restricts focus to
%% those.

%% The PDO is obviously finite if \props\ is, \eg\ if only atomic
%% properties are considered. In the presence of an infinite set of
%% composed plan properties (such as propositional formulas), the PDO may
%% still be finite due to considering equivalence classes. In particular,
%% the latter is necessarily the case if the set of plans \plans\ of
%% interest is finite (as is the case \eg\ if the action set is finite
%% and plan size can be finitely bounded), because then there is only a
%% finite number of non-equivalent plan properties.
%% %
%% When the PDO is finite, minimal elements exist and thus the PDA is
%% non-empty.

%% Regarding complexity, if \plans\ is implicitly given, then testing
%% plan-property satisfiability encompasses the plan existence problem
%% even for extremely simple plan properties (\eg, asking whether the
%% plan achieves some fact $p$). Similarly for plan-property entailments
%% $\entails{\plans}{\prop}{\propq}$, proving which corresponds to
%% proving a planning task unsolvable (there exists no plan that
%% satisfies \prop\ but not \propq). Tractable cases are limited to
%% explicitly given \plans, and planning fragments where \plans\ -- the
%% set of all plans -- can be generated in polynomial time. This is
%% exacerbated by the potentially exponential size of the PDO, and the
%% PDA, for composed plan properties. Certainly, the PDO/PDA should
%% ideally be computed offline, prior to interaction with a user. As our
%% experimental results show, however, in a variety of benchmarks,
%% computing a cPDA has similar scalability as optimal planning, and the
%% size of a cPDA is reasonably small for human inspection.



