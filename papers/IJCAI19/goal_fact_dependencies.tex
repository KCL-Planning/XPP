\section{Goal Dependencies in Oversubscription Planning}

\joerg{0.75 -- 1 page(s) joerg}

\joerg{To formulate the below as an instantiation of the framework:}
assume soft-goal fact set $G$. formulate all property sets as formulas
over $G$, which become atomic plan properties $\props^{G}$ by
evaluating them at end of plan. three different plan-property sets,
all with atomic properties $\props^{G}$ but with different composed
properties: 1. conjunctive properties $\props^{C}$, $\bigwedge_{g \in
  A} a$; 2. conjunctive exclusion properties $\props^{CE}$,
$\bigwedge_{g \in A} g \implies \neg \bigwedge_{g \in B} g$;
3. disjunctive exclusion properties $\props^{DE}$, $\bigwedge_{g \in
  A} g \implies \neg \bigvee_{g \in B} g$. Describe computation of PDA
for $\props = \props^{G} \cup \props^{C}$. Prove that the PDA for
$\props^{G} \cup \props^{CE}$ and $\props^{G} \cup \props^{DE}$ are
polynomially derivable from that (by specifying the required
derivation functions $f$ and proving them correct). 

\joerg{question: do we actually need these different versions later on
  in the paper? or should we only introduce 1.  formally, and merely
  briefly discuss the potential application to 2. and 3.? if so, we
  could presumably also remove the entire discussion of derivability?
  which seems rather a side line here (something we could save for a
  journal version)?}


\subsection{???}

\rebecca{start with a concrete example, in the truck domain}

\emph{If goal subset A is true at the end of the plan, then at least one element of goal subset B
must be false at the end of the plan.}

\rebecca{if we are looking fore the minimal sets $A \cup B$ than B always contains 
only one element}

\begin{definition}[Conjunctive Exclusion]
Given a planning task $\Pi = (V,A,c,I,G)$ $(A,B)$ with $A, B \subseteq G$ is a 
GFD1 if $\Pi$ with the goal $A \cup B$, i.e., $\bigwedge_{p \in A \cup B} p$
is unsolvable.
The subsumption relation is given by $\forall A,B,A',B': (A,B) \leq (A',B')$ iff $A \cup B \subseteq 
A' \cup B'$
\end{definition}

\paragraph{Algorithm}
All minimal GFD1s can be computed through a search tree that starts at node $N_0$ containing
all goal facts $G$ and where each search step on a node $N_i$ tests solvability of 
$\Pi_i = (V,A,c,I,N_i)$. If $\Pi_i$ is not solvable, we generate one child 
node $N'$ for every subset of $N_i$ obtained by removing one fact.  
Upon termination all nodes with only solvable children are the \emph{minimal unsolvable 
goal subsets (MUGS)}. (A,B) is a minimal GFD1 iff $A \cup B \in \text{MUGS}$, 
$A \cap B = \emptyset$ and $|B| = 1$

\rebecca{A is always solvable}

\rebecca{add example, for fuel level 5}

\subsection{???}

\textit{If goal subset A is true at the end of the plan, 
then ALL elements of goal subset B must be false at the end of the plan.}\\

\begin{definition}[Disjunctive Exclusion]
	Given a planning task $\Pi = (V,A,c,I,G)$ the tuples
	$(A,B)$ with $A,B \subseteq G $ is a GFD2 if 
	$\Pi$ with the goal $\bigwedge_{p \in A} p \wedge (\bigvee_{q \in B} q)$
	is unsolvable. 
	The subsumption relation is given by $\forall A,B,A',B': (A,B) \leq (A',B')$ iff $A \subseteq A'$
	and $B \supseteq B'$.
\end{definition}	

	\noindent
	Given all minimal GFD1s for $\Pi$ all minimal GFD2s can be derived according 
	to the following relation.

	\rebecca{proof}

	\vspace{-0.3cm}
	\begin{align*}
		GFD2 &:= \{(A,B) | 
				\exists P \in GFD1:(
				   A \subseteq P \wedge |P \setminus A | = 1 \wedge\\
				   &\forall P' \in PPD1:
					  A \subseteq P' \rightarrow P' \setminus A \subseteq B
				)
			 \}
	\end{align*}

\paragraph{Algorithm}
From the minimal GFD1s we get a set:
	$D = \{(A,B) | \exists P \in GFD1s, p \in P: A = P \setminus p \wedge
	B = \{p\}\}$

These GFD2s are not necessary optimal, the B could be larger.

\rebecca{find a name for A and B}

\noindent
To get the maximal set of goals which can not be achieved if we achieve A, 
you have to merge all B's which belong to an A' which is a subset of A. 

\begin{align*}
	(A, \bigcup_{(A', B') \in \{(A'', B'') \in D | A' \subseteq A\}} B' \cup B)
\end{align*}

\rebecca{to many ticks}

\rebecca{add example, from GFD1 to GFD2}

\noindent
If the planning task is not solvable for a goal fact at all, you can add 
this goal fact to all B's.\\
