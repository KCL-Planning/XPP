\documentclass{article}
\pdfpagewidth=8.5in
\pdfpageheight=11in

\usepackage{times}
\usepackage{helvet}
\usepackage{courier}
\usepackage{url}  %Required
\usepackage{graphicx}  %Required

%%%%%%%%%%%%%%%%%%%%%
%%%% Packages
\usepackage{amsthm,amsmath,amssymb}
\usepackage{tikz}
\usepackage{pgf-umlsd}
\usepackage{pgfplots}
\usepackage{pgfplotstable}
\usepgfplotslibrary{fillbetween}
\usepackage{ragged2e}
\usetikzlibrary{arrows,automata,fit,calc,positioning,shapes,shapes.multipart} %
%\usepackage[inline]{enumitem}
\usepackage{enumerate,url,soul}
\usepackage{paralist}
%\usepackage[usenames]{color} % Only used in comment commands
\usepackage[ruled,vlined]{algorithm2e}
\usepackage{thmtools,thm-restate}
\usepackage{listings}
\usepackage{subcaption}


%definition environment
%\newtheorem{mydef}{Definition}[chapter]
%\newtheorem{myexp}{Example}[chapter]
%\newtheorem{mytheo}{Theorem}[chapter]

\newcolumntype{L}[1]{>{\raggedright\arraybackslash}p{#1}}
\newcolumntype{C}[1]{>{\centering\arraybackslash}p{#1}}
\newcolumntype{R}[1]{>{\raggedleft\arraybackslash}p{#1}}

\definecolor{mygreen}{HTML}{009933}

\newcommand{\abseq}{/\hspace{-0.18cm}\sim^\alpha}

\newcommand{\titledate}[2][2.5in]{%
  \noindent%
  \begin{tabular}{@{}p{#1}@{}}
    \\ \hline \\[-.75\normalbaselineskip]
    #2
  \end{tabular}
}

\renewcommand{\textfraction}{0.05}


\begin{document}
\section{Action Set Properties}

\subsection{Syntax}

To definition of an action set property consists of two parts. The definition of 
the action subsets and the definition of the logical formula over the atoms 
\emph{at least one action in set X has been applied}.

\paragraph{Action Subsets}

The action set definition starts with the keyword \texttt{set} followed by a unique
name for the set and the number of actions contained in it.
The \texttt{<list of actions>} consists of one action definition per line.
An action definition consists of the action name followed by its arguments. 
All parts are separated with one space. 
The action parameters can either be object names or types. 
A "type-parameter" is a wild card and instantiated with all objects with the given type.

\begin{lstlisting}
set <name> <number of actions>
<list of actions>
\end{lstlisting}

\noindent
\textbf{Example}

\begin{lstlisting}
set delwithT0 4
unload p0 t0 location
load p1 t0 location
unload p1 t0 location
load p1 t0 location

set delwithT1 4
unload p0 t1 location
load p0 t1 location
unload p1 t1 location
load p1 t1 location
\end{lstlisting}

\paragraph{Property}

A property definition starts with the keyword \texttt{[soft-]property} followed 
by its unique name.
In the next line a logical formula over the beforehand defined set names defines 
the meaning of the property. Thereby the set names mean: \emph{at least one action in the set 
has been executed}.


The formula has to be in disjunctive normal form and in prefix notation.
- not \texttt{!}
- and \texttt{\&\&}
- or \texttt{||}

\begin{lstlisting}
[soft-]property <name>
<DNF>
\end{lstlisting}

A "soft-property" is included in the goal-fact dependency search while a "property" 
has to be always satisfied.\\

\noindent
\textbf{Example}


\begin{lstlisting}
soft-property same_truck
|| && delwithT0 ! delwithT1 && ! delwithT0 delwithT1
\end{lstlisting}


\paragraph{Propertyclass}
You can also define property classes with reduce "copy and paste"

\begin{lstlisting}
propertyclass same_truck(px,py)
{
set delwithT0 4
unload px t0 location
load px t0 location
unload py t0 location
load py t0 location

set delwithT1 4
unload px t1 location
load px t1 location
unload py t1 location
load py t1 location

soft-property same_truck
|| && delwithT0 ! delwithT1 && ! delwithT0 delwithT1
}
{
same_truck(p0, p1)
same_truck(p0, p2)
same_truck(p0, p3)
same_truck(p1, p2)
same_truck(p1, p3)
same_truck(p2, p3)
}
\end{lstlisting}


\paragraph{Soft Goals}
You can also declare some of the original goal facts to be soft goals at the end of the property file:

\begin{lstlisting}
soft-goals
<list of goal facts>
\end{lstlisting}

\noindent
\textbf{Example:}

\begin{lstlisting}
soft-goals
at(p0, l5)
\end{lstlisting}

\subsection{Compilation}

\paragraph{Variables}

\begin{itemize}
\item
For each set a new binary variable with the name \texttt{used\_<setname>} is added to the planning task. 
Initially this variables are 0. Every action in the set assigns the correponding variable to 1 in its effect. 
Therefore \texttt{used\_<setname> = 0} means: no action in the set has been executed and \texttt{used\_<setname>=1}
means at least one action in the set has been executed.

\item 
For each property a new binary variable with the name \texttt{sat\_<propertyname>} is added to the planning task.
Initially this variables are 0. In the evaluation phase the variables are assigned 1 if the corresponding 
property is satisfied.

\item 
An additional binary variable \texttt{eval\_phase} is added to indicate the phase of the plan, either execution 
or evaluation phase. Initially the variable is 0 which corresponds to execution phase.
\end{itemize}

\paragraph{Actions}

\begin{itemize}
\item
	To the precondition of every original action the condition \texttt{eval\_phase = 0} such that that
	they can only be executed in the execution phase.

\item
	To the effect of every original action the fact \texttt{"used\_setx" = 1} is added, if the action
	is contained in the set \texttt{setx}.

\item 
	To change from the evaluation to the execution phase an new action \texttt{change\_phase} with the
	precondition \texttt{eval\_phase = 0} and effect \texttt{eval\_phase = 1}.

\item 
	For each clause in the DNF of the property definition one action is added to the planning task.
	This action has the literals in the clause as preconditions and in its effect \texttt{sat\_propertynam = 1}. 
	All these actions have as additional precondition \texttt{eval\_phase = 1}. This ensures that the 
	property is first evaluated after all original actions are executed. 
\end{itemize}

\section{Goal Fact Relation Search}
TODO

\subsection{FD Output}
The output first shows the soft and hard goals.
Then all unsolvable minimal sub-goals are listed (only the soft goal part)


Example:

\begin{lstlisting}
*********************************
Size of tree: 128
Hard goals: 
Atom at(p3, l3)
Atom at(p2, l5)
Atom at(p1, l2)
Soft goals: 
soft_Atom at(p0, l5)
soft_sat_same_truck_p0_p1
soft_sat_same_truck_p0_p2
soft_sat_same_truck_p0_p3
soft_sat_same_truck_p1_p2
soft_sat_same_truck_p1_p3
soft_sat_same_truck_p2_p3
*********************************
Unsolvable:
soft_sat_same_truck_p2_p3
Unsolvable:
soft_sat_same_truck_p1_p2
Unsolvable:
soft_sat_same_truck_p0_p2|soft_sat_same_truck_p1_p3
Unsolvable:
soft_sat_same_truck_p0_p2|soft_sat_same_truck_p0_p3
Unsolvable:
soft_Atom at(p0, l5)|soft_sat_same_truck_p1_p3
Unsolvable:
soft_Atom at(p0, l5)|soft_sat_same_truck_p0_p3
Unsolvable:
soft_Atom at(p0, l5)|soft_sat_same_truck_p0_p1
*********************************
Number of minimal unsolvable goal subsets: 7
*********************************
\end{lstlisting}


\end{document}


